\documentclass[a4paper,11pt]{article}

\usepackage[T1]{fontenc}
\usepackage[utf8]{inputenc}
\usepackage{graphicx}
\usepackage{xcolor}

\renewcommand\familydefault{\sfdefault}
\usepackage[defaultmono]{droidmono}

\usepackage{enumerate}
\usepackage{hyperref} 

\usepackage{geometry}
\geometry{total={210mm,297mm},
left=25mm,right=25mm,%
bindingoffset=0mm, top=20mm,bottom=20mm}


\linespread{1.3}

\newcommand{\linia}{\rule{\linewidth}{0.5pt}}
\renewcommand{\arraystretch}{1.5}
\makeatletter
\renewcommand{\maketitle}{
\begin{center}
\vspace{2ex}
{\huge \textsc{\@title}}
\vspace{1ex}
\\
\linia\\
\@author \hfill \@date
\vspace{4ex}
\end{center}
}
\makeatother

\usepackage{fancyhdr}
\pagestyle{fancy}
\lhead{}
\chead{}
\rhead{}
\lfoot{ChallP FS16}
\cfoot{}
\rfoot{Page \thepage}
\renewcommand{\headrulewidth}{0pt}
\renewcommand{\footrulewidth}{0pt}
%

\begin{document}

\title{GPIO Factsheet}

\author{fbinna, vmeier, laquino}

\date{2016}

\maketitle
\part{Beschreibung GPIO}
\section*{Facts}
\begin{tabular}{| p{3.5cm} | p{10cm} |}
	\hline
	\textbf{Abkürzung} & General Purpose Input/Output\\\hline
	\textbf{Übertragungsraten} & -\\\hline
	\textbf{Anschluss} & -\\\hline
	\textbf{Topologie} & -\\\hline
	\textbf{Synchron/Asynchron} & -\\\hline
	\textbf{Fehlervermeidung} & -\\\hline
	\textbf{Adressierung} & -\\\hline
	\textbf{Weitere Geräte} & 
	\begin{itemize}
		\item LEDs
		\item Buzzer
	\end{itemize}
	\\\hline
\end{tabular}
\section{Generelle Beschreibung}
GPIO steht für "General Purpose Input/Output" und es handelt sich dabei um einen Pin auf einem Integrierten Schaltkreis, welcher keiner bestimmten Funktion zugeordnet ist, sondern vom Benutzer des Geräts nach Wünschen eingesetzt werden kann. Als GPIO Port wird eine Menge von typischerweise 8 Pins bezeichnet, welche als Gruppe kontrolliert werden. Ports oder einzelne PINs können als Input oder Output konfiguriert werden und disabled beziehungsweise enabled werden. \\
Ein Pin kann die Werte 1 (High) oder 0 (Low) entsprechend der Spannung annehmen. Eine Port von 8 Pins kann also Werte von 0 (00000000) bis 255 (11111111) parallel übertragen.
	
\end{tabular}



\section*{Quellen}
\href{https://en.wikipedia.org/wiki/General-purpose_input/output}{Wikipedia Artikel zu GPIO}\\
\href{http://www.mikrocontroller.net/articles/HD44780}{Microkontroller.net Beschreibung zu HD44780}\\

\part{Beschreibung HD44780 LCD Display}
\section{Register}

\section{Codebeispiel}


\end{document}
